
\documentclass{ctexbook}

\ctexset{
chapter/format  += \raggedright,
section/format  += \raggedright,
}

\usepackage{bcprules}
\usepackage{stmaryrd}            % more math symbols
\usepackage{amsmath}             % ams math
\usepackage{amsfonts}            % ams fonts
\usepackage{amssymb}             % amsmath
\usepackage{amsthm}              % theorem environment
\newtheorem{theorem}{{定理}}[chapter]
\newtheorem{lemma}{{引理}}[chapter]
\newtheorem{corollary}{{推论}}[chapter]
\newtheorem{definition}{{定义}}[chapter]



\usepackage{mathrsfs}            % mathrsfs
\usepackage[final]{listings}     %
\usepackage[bookmarks,colorlinks]{hyperref}
\usepackage{booktabs} % table
\usepackage{geometry}
\geometry{left=20mm, right=20mm, top=26mm, bottom=18mm}



%%%%%%%%%%%%%%%%%%%%%%%%%%%%%%%%%%%%%%%%%%%%%%%%%%%%%%%%%%%%%%%%%%%%%%%%%%%%%
%%  Latin
%%%%%%%%%%%%%%%%%%%%%%%%%%%%%%%%%%%%%%%%%%%%%%%%%%%%%%%%%%%%%%%%%%%%%%%%%%%%%
\newcommand\etal{\emph{et al.}}
\newcommand\eg{\emph{e.g.,\ }}
\newcommand\ie{\emph{i.e.,\ }}
\newcommand\etc{\emph{etc.}}

%%%%%%%%%%%%%%%%%%%%%%%%%%%%%%%%%%%%%%%%%%%%%%%%%%%%%%%%%%%%%%%%%%%%%%%%%%%%%
%%  Machine Syntax
%%%%%%%%%%%%%%%%%%%%%%%%%%%%%%%%%%%%%%%%%%%%%%%%%%%%%%%%%%%%%%%%%%%%%%%%%%%%%
\newcommand{\Comm}[1]{\mbox{\textit{#1}}}
\newcommand{\Insn}[1]{\mdsf{#1}}
\newcommand{\Reg}[1]{\mathtt{#1}}
\newcommand\Prog[0]{{\mathbb{P}}}
\newcommand\Mem[0]{{\mathbb{M}}}
\newcommand\Smem[0]{\mathbb{M}_{s}}
\newcommand\Wmem[0]{\mathbb{M}_{w}}
\newcommand\Wmk[1]{\MEM_{w#1}}
\newcommand\Rmk[1]{\MEM_{r#1}}
\newcommand\Heap[0]{{\mathbb{H}}}
\newcommand\Rheap[0]{\mathbb{H}_{r}}
\newcommand\Wheap[0]{\mathbb{H}_{w}}
\newcommand\Regs[0]{{\mathbb{R}}}
\newcommand\Rr[0]{\Reg{r}}
\newcommand\Iseq[0]{{\mathbb{I}}}
\newcommand\Code[0]{{\mathbb{C}}}
\newcommand\Thread[0]{{\mathbb{T}}}
\newcommand\State[0]{{\mathbb{S}}}
\newcommand\Xstate[0]{{\mathbb{X}}}
\newcommand\Bstate[0]{{\mathbb{B}}}
\newcommand\Status[0]{{\mathbb{A}}}
\newcommand\Ths[0]{\mathbb{TS}}
\newcommand\Tid[0]{\mathtt{t}}

\newcommand\Ll[0]{{\mathtt{l}}}
\newcommand\Lf[0]{{\mathtt{f}}}
\newcommand\Pc[0]{\mathtt{pc}}
\newcommand\Pcinc[0]{\mathtt{pc+1}}
\newcommand\Word[0]{{\mathtt{w}}}
\newcommand\Value[0]{{\mathtt{v}}}
\newcommand\Rrd[0]{\Rr_d}                        % Destination Register
\newcommand\Rrs[0]{\Rr_s}                        % Source Register
\newcommand\Rrt[0]{\Rr_t}                        % Secondary Source Register

\newcommand\Instr[0]{\iota}
\newcommand{\nmm}[1]{\mbox{\choosernsize{\tiny}{\small}\uppercase{#1}}}
\newcommand\interp[1]{[\hspace{-1pt}[#1]\hspace{-1pt}]}
\newcommand\CAPmapsto[0]{\leadsto}
\newcommand{\Arrow}{\!\rightarrow\!}
\newcommand{\Parrow}{\!\rightharpoonup\!}
\newcommand{\Drrow}{\!\leftrightarrow\!}
\newcommand{\Land}{\land\hspace{-6.5pt}\land}
\newcommand{\Lor}{\lor\hspace{-6.5pt}\lor}
\newcommand{\Neg}{\neg\hspace{-4.5pt}\neg}
\newcommand{\Larrow}{\!\Rightarrow\!}
\newcommand{\Ldarrow}{\!\Leftrightarrow\!}
\newcommand{\Arrowimplam}{\stackrel{\lambda}{\Rightarrow}}
\newcommand{\Arroweqv}{\Leftrightarrow}
\newcommand{\Arrowall}{\stackrel{\forall}{\Rightarrow}}
\newcommand{\Fog}[2]{#1\circ#2}
\newcommand{\Equiv}{\!\triangleq\!}
\newcommand{\Judge}[2]{#1\vdash#2}
\newcommand{\Indu}[2]{#1^{#2}}
\newcommand{\Wand}{\!-\!\!*\,}
\newcommand{\Disd}{\!\perp\!}
\newcommand{\Disu}{\!\uplus\!}
\newcommand{\Oftype}[2]{#1\!:\!#2}

\newcommand\NS[0]{\kw{Next}_{(\Pc,\iota)}}
\newcommand\NEXTS[1]{\NS(#1)}
\newcommand\NEXTPC[1]{\kw{Next}_{\kw{pc}}(#1)}
\newcommand\NPC[0]{\kw{Npc}_{(\Pc,\iota)}}

\newcommand\K[0]{\CAPfont{k}}
\newcommand\ONETON{\{1,\ldots, n\}}

\newcommand\step[0]{\longmapsto}
\newcommand\tstep[1]{\stackrel{#1}\step}

\newcommand\Cspec[0]{\psi}
\newcommand\Pspec[0]{\phi}
\newcommand\Inv[0]{\kw{INV}}

\newcommand\ASST[0]{\mathtt{a}}
\newcommand\Pred[0]{\mathtt{p}}
\newcommand\Post[0]{\mathtt{q}}
\newcommand\Guar[0]{\mathtt{g}}
\newcommand\MPred[0]{\mathtt{m}}
\newcommand\WMPred[0]{\ASST_{w}}
\newcommand\RMPred[0]{\ASST_{r}}
\newcommand\WP[1]{\ASST_{w#1}}
\newcommand\RP[1]{\ASST_{r#1}}

\newcommand\DEFEQ{\stackrel{\mathrm{def}}{=}}
\newcommand\empH[0]{\kw{emp}}
\newcommand\PrecPred[1]{\kw{Precise}(#1)}

\newcommand\Domain[1]{\kw{dom}({#1})}
\newcommand\RLKPred[1]{\kw{rlk}~\LOCK~#1}
\newcommand\URLKPred[1]{\kw{unrlk}~\LOCK~#1}
\newcommand\WLKPred[1]{\kw{wlk}~\LOCK~#1}
\newcommand\UWLKPred[1]{\kw{unwlk}~\LOCK~#1}

\newcommand\septrac[0]{-\!\!\!\circledast\;}%septraction
\newcommand\mwand[0]{-\!\!\!*\;}
\newcommand\MPredI[1]{\ASST_{\LOCKSPEC}^{#1}}
\newcommand\enable[1]{\kw{En}~(#1)}


\title{FPFL}
\begin{document}

\chapter{翻译须知}

\section{正文}

\textbf{标点符号}

译稿中的标点符号要遵循中文表达习惯和中文标点符号的使用习惯,不能照搬原文的标点符号。

使用中文标点,对比中文英文括号:
\begin{itemize}
    \item (英文)():!?,.
    \item (中文)():!?,。
\end{itemize}

\textbf{字体}

\vspace{4mm}
\begin{tabular}{ll}
    \toprule
    pdf                          &  latex code                         \\
    \midrule
    正文为宋体                   &  \verb!正文为宋体!                  \\
    \textit{英文斜体对应楷体}    &  \verb!\textit{英文斜体对应楷体}!   \\
    \textbf{加粗对应黑体}        &  \verb!\textbf{加粗对应黑体}!       \\
    \texttt{等宽字体对应仿宋}    &  \verb!\texttt{等宽字体对应仿宋}!   \\
    \bottomrule
\end{tabular}
\vspace{4mm}

关于英文术语的表述。英文术语首次出现时,应该根据该术语的流行情况,优先使用简写形式,
并在其后使用括号加英文、中文全称注解,格式为(举例):HTML(Hypertext
Markup Language,超文本标识语言)。
然后在下文中直接使用简写形式。当然,必要时也可以根据语境使用中、英文全称。

\paragraph{参考}

\begin{itemize}
\item \href{http://www.cmpbook.com/index.php?id=131}{机械工业出版社作译者须知}.
\item \href{http://www.ituring.com.cn/article/501527}{图灵技术图书译者须知}.
\end{itemize}

\section{数学}

更多使用的教程参考 \href{https://en.wikibooks.org/wiki/LaTeX/Mathematics}{\LaTeX/Mathematics wiki}

\subsection{公式编号}

行内数学环境: $a + b$ , \verb!$a + b$! 或者 \verb!\(a + b\)!。

多行数学环境使用 \verb!$$a + b$$!,  \verb!\[a + b\]!。
$$a + b$$

有编号的公式:

\begin{equation}\label{euqation:one_plus_one}
    1 + 1 = \lambda f . \lambda x . f\, x
\end{equation}


子编号:
\begin{subequations}
    \begin{eqnarray}
        a = b \\
        c = d
    \end{eqnarray}
\end{subequations}

如果不需要编号,在环境名后面加上星号即可: \verb!\begin{equation*} \end{equation*}!

\subsection{字体}

数学字体

\begin{tabular}{ll}
$\mathrm{JABCDEabcde1234}$      &  \verb!$\mathrm{JABCDEabcde1234}$!     \\
$\mathit{JABCDEabcde1234}$      &  \verb!$\mathit{JABCDEabcde1234}$!     \\
$\mathsf{JABCDEabcde1234}$      &  \verb!$\mathsf{JABCDEabcde1234}$!     \\
$\mathtt{JABCDEabcde1234}$      &  \verb!$\mathtt{JABCDEabcde1234}$!     \\
$\mathnormal{JABCDEabcde1234}$  &  \verb!$\mathnormal{JABCDEabcde1234}$! \\
$\mathcal{JABCDEabcde1234}$     &  \verb!$\mathcal{JABCDEabcde1234}$!    \\
$\mathscr{JABCDEabcde1234}$     &  \verb!$\mathscr{JABCDEabcde1234}$!    \\
$\mathfrak{JABCDEabcde1234}$    &  \verb!$\mathfrak{JABCDEabcde1234}$!   \\
$\mathbb{JABCDEabcde1234}$      &  \verb!$\mathbb{JABCDEabcde1234}$!     \\
\end{tabular}

如果在数学环境的中途需要使用文字, 使用 \verb!\text{}!.

\subsection{定理与证明}

\begin{theorem}[稳定性]\label{theorem:stability}
 如果 $\Gamma \vdash_{\mathcal{R}} J$ 则 $\Gamma \vdash_{\mathcal{R} \cup \mathcal{R}'} J$
\end{theorem}

(找不到原书使用的 J 是是什么字体, 可能是 J 去掉了前面的勾)

在 \texttt{local.tex} 中定义了定理(\texttt{theorem})、引理(\texttt{lemma})、
推论(\texttt{corollary})、定义(\texttt{definition})的环境。请使用这些环境。


证明需要使用 \texttt{proof} 环境,在证明结束后会有一个方框。
\begin{proof}
    显然。
\end{proof}


\subsection{宏}

\LaTeX 可以使用 \verb!\newcommand\commandname{commandbody}! 来定义新的命令。
如果 commandname 已经被定义,使用 \verb!renewcommand! 覆盖原有定义。

\texttt{local.tex} 文件中定义了一些宏。

\begin{verbatim}
\newcommand{\Comm}[1]{\mbox{\textit{#1}}}
\end{verbatim}

\texttt{[1]} 表示接受一个参数, \texttt{\#1}用来引用这个参数. 参数通过 \texttt{\{\}} 传递。

例如: $\Comm{add}$  $\qquad$ \vspace{1cm} \verb!$\Comm{add}$!

可以在文中随时定义新的宏,在使用后删除宏,避免污染环境。

\newcommand{\aaaa}{aaaa}
\aaaa
\let\aaaa\undefined

在 \texttt{bcprules.sty} 文件中定义了用于排版 inference rules 的宏。


\section{引用}

在需要引用的公式,章节,定理后面加上 \verb!\label{label_name}!, 使用 \verb!\ref{label_name}!
引用。名字最好有意义。涉及章节的引用先空着,用注释说明。

定理\ref{theorem:stability}, 公式 \ref{euqation:one_plus_one}。


\section{索引}

暂时不用考虑。



\end{document}
